% !TeX spellcheck = en_US
% !TeX root = ../build/intro-proving-system.tex
% !TeX TXS-program:compile = txs:///xelatex/[--shell-escape]




\section{Selector Columns}

Until now, to illustrate the behavior of the general executor we have been talking about the \textit{k-th row performing an instruction} \texttt{OPx}. But since we are using constraints with the entire columns and not individual cells (recall that we are actually interpolating), we need to add \textbf{selector columns} in order to indicate which operation we are performing in each row and reflect it within constraints. In other words, we have to ensure that the set of constraints is fulfilled in \textbf{every row}. Selector columns are used to control whether the constraints of an operation apply or not, meaning whether in a specific row we apply \texttt{OPx} or \texttt{OPy}. For example, we add two selector columns \texttt{OP1} and \texttt{OP2} into the previous example as shown in Figure \ref{fig:selector-columns}.
\begin{align*}
&\texttt{OP1}: a'=a+b+c \\
&\texttt{OP2}: a'=a+b+c+b'+c'
\end{align*}

\begin{figure}[H]
\centering
\begin{tabular}{|c|c|c|c|c|}\hline
\centering
$\A$ & $\B$ & $\C$ & $\mathtt{OP1}$ & $\mathtt{OP2}$ \\ \hline
$a_0$ & $b_0$ & $c_0$ & 1 & 0 \\ \hline
$a_1$ & $b_1$ & $c_1$ & 0 & 1\\ \hline
$a_2$ & $b_2$ & $c_2$ & & \\ \hline
\end{tabular}
\caption{Columns \texttt{OP1} and \texttt{OP2} flag whenever one of these two operations is being performed. }
\label{fig:selector-columns}
\end{figure}

But now, we need to introduce this columns into the constraints so that they become satisfied in every row. In this case, we introduce the following constraint:
\[
op1 \cdot (1-op2) \cdot (a+b+c-a') + op2 \cdot (1-op1) \cdot (a+b+c+a'+b'-c') = 0
\]

\begin{bremark}
Observe that, at this moment, constraints are not \textbf{instruction related} anymore. In previous sections, we constraint each of the operations separately, which can not be proved with any known proving system directly.
\end{bremark}

We can check that if we are in the first row, then $op1=1$ and $op2=0$, so the constrain becomes:
$$1 \cdot (1-0) \cdot (a+b+c-a') + 0 \cdot (1-1) \cdot (a+b+c+a'+b'-c') = \boxed{a+b+c-a'=0},$$
which corresponds to performing \texttt{OP1} in te first row. Same can be checked for \texttt{OP2}. Note that this selector columns do not depend on the inputs but on the computation itself, so they can be preprocessed.
