% !TeX spellcheck = en_US
% !TeX root = ../build/intro-proving-system.tex
% !TeX TXS-program:compile = txs:///xelatex/[--shell-escape]




\section{Secondary Execution Trace Matrices and Lookup Arguments}

\subsection{Introduction}

Let’s assume that our cryptographic backend only allows to define constraints with additions and multiplications. Let’s consider also that we want to implement a exponentiation operation (which we will denote by \texttt{EXP}). Then, we can use several rows doing multiplications to implement \texttt{EXP}. A portion of the execution trace (that implements the operation $2^5 = 32$) is shown in Figure \ref{fig:exp-example}.

\begin{figure}[H]
\centering
\begin{tabular}{|c|c|c|}
\hline
\A & \B & \C \\ \hline
\cellcolor{cyan} 2 & \cellcolor{cyan} 5 & 2  \\ \hline
2 & 4 & 4  \\ \hline
2 & 3 & 8  \\ \hline
2 & 2 & 16 \\ \hline
2 & 1 & \cellcolor{green} 32 \\ \hline
\end{tabular}
\caption{Execution trace modeling the exponentiation operation $2^5$. Blue cells represent the inputs and the green cell the output of the  computation. }
\label{fig:exp-example}
\end{figure}

Where the columns \A, \B and \C have the following constraints between them:
\begin{enumerate}
\small
\item $a'=a$, the \A column represents the base.
\item $b'=b-1$, the \B column stores the decreasing exponent.
\item $c'=c\cdot a$, the \C column stores the intermediate results.
\end{enumerate}


When implementing this operation in the main execution trace, it's important to know that each \texttt{EXP} operation will consume multiple rows. In fact, The exact number of rows required by \texttt{EXP} varies depending on the exponent. This approach leads to a very complicated set of constraints together with a huge amount of consumed rows by operation. To solve this, we can take another approach in using \textbf{tailor-made secondary execution traces} for specific operations.

In this approach, there is a \textbf{main execution trace} and there are also \textbf{secondary execution traces}. In the cryptographic back-end, we use a mechanism called lookup argument to link these execution trace matrices. In particular, the lookup argument provides the constraints necessary to check that
certain cells of a row in an execution trace matrix match other cells in a row of another execution trace matrix.

\subsection{Linking Execution Traces via Lookup Arguments}

Let us continue with the \texttt{EXP} example to show how the several execution traces approach works. We will have \textbf{two execution traces}. In the \textbf{main execution trace}, shown in Figure \ref{fig:main-sm-exp-example}, each \texttt{EXP} operation will occupy just one row. For the first EXP operation, inputs are $2$ and $5$ and the result is $32$. Observe that the column \texttt{EXP} flags whenever an \texttt{EXP} operation is performed. In such rows, we should ensure that the value of the column \C precisely corresponds to the result of exponentiating the value in column \A to the value in column \B. The correctness of the result will be validated in a \textbf{secondary execution matrix}.

\begin{figure}[ht]
\centering
\begin{tabular}{|c|c|c|c|}
\hline
\A &  \B & \C &  \texttt{EXP}\\ \hline
$\cdots$ & $\cdots$ & $\cdots$ & 0 \\ \hline
2 & 5 & 32 & 1 \\ \hline
$\cdots$ & $\cdots$ & $\cdots$ & 0 \\ \hline
$\cdots$ & $\cdots$ & $\cdots$ & 0 \\ \hline
$\cdots$ & $\cdots$ & $\cdots$ & 0 \\ \hline
$\cdots$ & $\cdots$ & $\cdots$ & 0 \\ \hline
3 & 2 & 9 & 1 \\ \hline
$\cdots$ & $\cdots$ & $\cdots$ & 0 \\ \hline
$\cdots$ & $\cdots$ & $\cdots$ & 0 \\ \hline
& & & \\ \hline
\end{tabular}
\caption{Depiction of the main execution trace. When \texttt{EXP} column is equal to $1$ means that we should check that the value of \C can be derived by exponentiating the value in column \A to the power of the value in column \B. To do so, we will use a secondary execution trace. }
\label{fig:main-sm-exp-example}
\end{figure}

\begin{bremark}
An execution trace matrix can be seen as a set of states \textbf{(state machine)}, in which each row is a state. We will use both, the
terms \textbf{execution trace matrix} and \textbf{state machine} interchangeably.
\end{bremark}

The secondary state machine will operate very similar as the one shown before in Figure \ref{fig:exp-example}. In this case, the columns have the following sense:
\begin{enumerate}
\small
\item The \A column represents the base. Observe that this column is allowed to contain several values as we may check several \texttt{EXP} operations in the same state machine as long as we do not consume more rows than the fixed maximum.

\item The \B column stores the exponent.

\item The \C column stores the intermediate results.

\item The column \D stores a decreasing counter. In each \texttt{EXP} operation, it starts with the same value as the exponent and keeps decreasing until arriving $1$, meaning that the current operation has been finished.

\item The column \texttt{EXP} flags whenever the operation is actually finished. Rows with \texttt{EXP} equal to $1$ are the ones where we can state consistency of \texttt{EXP} operations of the Main State Machine via a lookup argument.
\end{enumerate}

The connection between the two state machines will be established through a lookup argument. Whenever \texttt{EXP} selector's value in the Main State Machine is $1$, it will be searched whether the values in the columns \A, \B and \C of the same row, are present in the columns \A, \B and \C respectively of the Secondary State Machine in a row also having \texttt{EXP} selector equal to $1$. Affirmative response will mean that the \texttt{EXP} operation in the Main State Machine is performed correctly and, otherwise, that the prover has cheated.

\begin{figure}[ht]
\begin{minipage}[t]{0.5\linewidth}
\centering
\textbf{Main State Machine} \\
\vspace{1em}
\begin{tabular}{|c|c|c|c|}
\hline
\A &  \B & \C &  \texttt{EXP}\\
\hline
$\cdots$ & $\cdots$ & $\cdots$ & 0 \\ \hline
2  \cellcolor{yellow} & 5 \cellcolor{yellow} &
32 \cellcolor{yellow} & 1 \cellcolor{yellow} \\
\hline
$\cdots$ & $\cdots$ & $\cdots$ & $\cdots$ \\ \hline
3  \cellcolor{yellow} & 2 \cellcolor{yellow} & 9
\cellcolor{yellow} & 1 \cellcolor{yellow} \\ \hline
$\cdots$ & $\cdots$ & $\cdots$ & 0 \\ \hline
\end{tabular}
\vspace{1em}
\begin{flushleft} \hspace{1.5cm}
$ \begin{array}{|c|}
  \hline \cellcolor{yellow} \\ \hline
\end{array}$
: Lookup
\end{flushleft}
\end{minipage}
\hspace{-3em}
\begin{minipage}[t]{0.5\linewidth}
\centering
\textbf{\texttt{EXP} Secondary State Machine} \\
\vspace{1em}
\begin{tabular}{|c|c|c|c|c|}
\hline
\A & \B & \C & \D
& \texttt{EXP} \\ \hline
2 & 5 & 2 & 5 & 0 \\ \hline
2 & 5 & 4 & 4 & 0 \\ \hline
2 & 5 & 8 & 3 & 0 \\ \hline
2 & 5 & 16 & 2 & 0 \\ \hline
2 \cellcolor{yellow} & 5 \cellcolor{yellow} &
32 \cellcolor{yellow} & 1 & 1
\cellcolor{yellow}\\ \hline
3 & 2 & 3 & 2 & 0 \\ \hline
3 \cellcolor{yellow} & 2 \cellcolor{yellow} & 9
\cellcolor{yellow} & 1 & 1 \cellcolor{yellow} \\ \hline
& & & & \\ \hline
\end{tabular}
\end{minipage}

\caption{The linkage between the Main State Machine and the \texttt{EXP} State Machine is done by checking equality between yellow rows of both state machines. In the Main SM, we just put the inputs/outputs, that we call \textit{free}, in a single row.}
\label{fig:lookup-exp}

\end{figure}

Recall that there is a PIL compiler that reads a PIL specification file and compiles it to an output file with the list of constraints and a format that can be consumed by the prover to generate the proof. In the PIL language, the state machines, that is, the different subexecution matrices linked together via lookup arguments, are called \texttt{namespaces}. In the \href{https://github.com/0xPolygonHermez/zkevm-proverjs}{\texttt{zkevm-proverjs}} repository, you can find the PIL specification of the whole zkEVM. There are several files, one for each different state machine, such that \texttt{binary.pil}, that is responsible for constraining \textbf{binary operations} (such that additions of $256$ numbers represented in base $2$ and equality/inequality comparators), or \texttt{mem.pil}, which is responsible for managing \textbf{memory-related opcodes}. The Main State Machine is constrained in the \texttt{main.pil}, which serves as the main entrypoint for the whole PIL of the zkEVM.

\subsection{Final Remarks and Future Improvements}

The columns of each state machine are defined by the design of its corresponding
execution trace. Due to constraints in our existing cryptographic backend, it is mandatory that all state machines share the same number of rows. The computation of an L2 batch can have branches and loops and hence, each L2 batch execution can use a different number of operations in the zkEVM.  Consequently, the number of rows utilized by each state machine depends on the specific operations carried out during batch execution. Since the number of rows is fixed (and the same for all state machines) we can have unused rows. But, what is more important is that obviously, the size of the computation being proved must fit in the execution trace matrices available.

\begin{figure}[ht]
\centering
\includegraphics[width=0.85\columnwidth]{\zkevmdir/figures/architecture/intro-proving-system/shape-state-machines}
\caption{Due to the specific cryptographic backend used in the prover, all the same state machines should have the same fixed amount of rows. }
\label{fig:several-sm-rows}

\end{figure}

Currently, we are in the process of developing a new version of PIL, referred to as \textbf{PIL2}. \textbf{PIL2} is designed to operate with a more potent cryptographic backend capable of generating an adequate number of subexecution traces to accommodate the whole batch processing without running out of rows. Additionally, we are collaborating with other projects within the \textit{zk projects} at Polygon to establish a standardized format for the PIL output file, named \texttt{pilout}.



\section{Recap of the Prover and the Verifier}

The Prover, shown in Figure \ref{fig:prover-diagram}, is the component within the Proving System which is in charge of generating a proof for the correct execution of some program. The program, which takes care of the computational aspect of the computation, is written in a language called \textbf{zkASM}, developed by the Polygon team. Providing also the private and the public inputs, the \textbf{Executor} component within the Prover is able to generate the execution traces designed to model the willing computation. The executor produces two binary files: one containing the \textbf{fixed columns}, which should only be generated once if we do not change the computation itself, and the other containing the \textbf{witness columns}, which vary with inputs and thus need to be generated anew for each proof. The files containing the pre-processed fixed columns and the processed witness columns for the zkEVM are temporary stored in binary files and are quite large, consisting on more than $100$Gb. Subsequently, the cryptographic backend of the prover, in conjunction with the compiled PIL constraints through \texttt{pilcom} and the execution trace binary files generated earlier, can produce the proof and provide the public values, both inputs and outputs, for the verifier.

\begin{figure}[H]
\centering
\includegraphics[width=0.85\columnwidth]{\zkevmdir/figures/architecture/intro-proving-system/prover-diagram.drawio}
\caption{Schema of the Prover component. The Prover component has $3$ subcomponents: the Executor, which is in charge of populating the execution trace based on some computation with values depending on some inputs, the PIL compiler or \texttt{pilcom}, which compiles PIL files into JSON files prepared to be consumed by the Prover, and the cryptographic backend, which takes the output of both the Executor and the \texttt{pilcom} and generates the proof and the publics for the verifier. }
\label{fig:prover-diagram}
\end{figure}

Posteriorly, the verifier (depicted in Figure \ref{fig:verifier-diagram}) use both the proof and the public values to check if the prover has performed the computation in a correct way.

\begin{figure}[H]
\centering
\includegraphics[width=0.65\columnwidth]{\zkevmdir/figures/architecture/intro-proving-system/verifier-diagram}
\caption{The verifier utilizes both the proof and the public values to validate the correctness of the computation performed by the prover.}
\label{fig:verifier-diagram}
\end{figure}





