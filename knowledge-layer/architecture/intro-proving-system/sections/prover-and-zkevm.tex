% !TeX spellcheck = en_US
% !TeX root = ../build/intro-proving-system.tex
% !TeX TXS-program:compile = txs:///xelatex/[--shell-escape]



\section{The Execution Trace and the zkEVM}

In our current cryptographic back-end, the dimension of the execution trace is predetermined. Additionally, Also, we don’t know exactly what EVM opcodes (and as a consequence zkEVM operations) will be executed, since this depends on the particular transactions of the L2 batch (See Figure \ref{fig:batch-evmopcode-zkop-trace}). The pre-fixed shape fixes in turn the amount of computation that we can do, which in our case is the amount and type of L2 transactions for which we can generate a proof. In general, it is hard to optimize the shape of a single execution trace matrix:

\begin{itemize}
\item \textbf{Narrow matrices may easily hit the max row limit}: Increasing the number of rows is delicate because, since the number of rows is a power of two, we can only double the size of the matrix, not add extra rows one by one. So we have to be careful when adding rows, making sure that (if possible) we do not double the size of the matrix unnecessarily.

\item\textbf{Wide matrices might be inefficient:} Wide matrices may have too many unused cells and, furthermore, they might be inefficient for mixing many different instructions. Moreover, adding more columns also increase proving time.
\end{itemize}


\begin{figure}[h]
\centering
\includegraphics[width=0.65\columnwidth]{\zkevmdir/figures/architecture/intro-proving-system/batch-evmopcode-zkop-trace.drawio}
\caption{Every transaction in a batch consists of multiple EVM Opcodes, and each EVM Opcode corresponds to multiple zkEVM Opcodes, the instructions in charge of filling the execution trace.}
\label{fig:batch-evmopcode-zkop-trace}
\end{figure}

