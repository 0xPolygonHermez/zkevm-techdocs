% !TeX spellcheck = en_US
% !TeX root = ../build/intro-proving-system.tex
% !TeX TXS-program:compile = txs:///xelatex/[--shell-escape]



%%%%%%%%%%%%%%%%%%%%%%%%%%%%%%%%%%%%%%%%%%%%%%%%%%%%%%%%%%%%%%%%%%%%%%%%%%%%%
\section{Introduction to the Prover}
The \textbf{Prover} is a software component whose main goal is to generate a proof that for the correct execution of a given program with an specific set of inputs. While the process of generating a proof is resource-consuming, the time required to verify a constructed proof is significantly shorter, enabling it to be done by a smart contract. The main usage of the prover in the zkEVM context, presented in Figure \ref{fig:goal-prover}, is to generate a proof $\pi_{i+1}$ stating that, given a batch of transactions ${Batch}_i$ and a L2 State $S_i^{L2_x}$, the state has transitioned to $S_{i+1}^{L2_x}$.

\begin{figure}[H]
\centering
\includegraphics[width=0.8\columnwidth]{\zkevmdir/figures/architecture/intro-proving-system/goal-prover.drawio}
\caption{The Prover component generates proof $\pi_{i+1}$ that state the accuracy of the state transition.}
\label{fig:goal-prover}
\end{figure}

To produce this proof, the initial step involves the creation of an execution matrix. An \textbf{execution matrix} (also called, \textbf{execution trace}) is a matrix that records all the intermediate computations constituting a larger computation. In the zkEVM context, we can think that the big computation is the state transition function and the little intermediate ones are the zkEVM instructions or opcodes. As we will note later on, a single EVM opcode may be implemented using one or more zkEVM opcodes, indicating a lack of a strict one-to-one correspondence between EVM opcodes and zkEVM ones.


\subsection{Execution matrix: A Toy Example}

Suppose that we want to model the following computation
\[
[(x_0+x_1)\cdot 4] \cdot x_2
\]
using an execution trace, being $x := (x_0, x_1, x_2)$ a given set of inputs. Also suppose that the only available instructions (or operations) are the ones described below:

\begin{enumerate}

\item Copy inputs into cells of the execution trace.

\item Sum two cells of the same row, and leave the result in a cell of the next row (referred to as the \texttt{ADD} instruction).

\item Multiply by a constant, and leave the result in a cell of the next row (referred to as the \texttt{TIMES4} instruction).

\item Multiply two cells of the same row, and leave the result in a cell of the next row (referred to as the \texttt{MUL} instruction).

\end{enumerate}

We will use a matrix with three \textbf{columns} \A, \B and \C. Also suppose that the number of rows is bounded by some constant $N$, which we will call \textbf{length} of the execution trace. We can depict it as in Figure \ref{fig:general-execution-trace}. The columns of an execution trace are often called \textbf{registers} (so we may name them interchangeably).

\begin{figure}[H]
\centering
\begin{tabular}{|c|c|c|}
\hline
\A & \B & \C \\ \hline
$a_0$ & $b_0$ & $c_0$   \\ \hline
$a_1$ & $b_1$ & $c_1$   \\ \hline
\dots & \dots & \dots   \\ \hline
$a_N$ & $b_N$ & $c_N$   \\ \hline
\end{tabular}
\caption{General form of an execution trace of length $N$ having three registers \A, \B and \C.}
\label{fig:general-execution-trace}
\end{figure}

Let's provide an specific example giving the following set of inputs $x=(1,2,5)$. We can model our desired computation
\[
[(x_0+x_1)\cdot 4] \cdot x_2 = [(1+2)\cdot 4] \cdot 5 = 60.
\]
to fit our execution trace using only the available operations as follows:
\vspace{0.3cm}
\begin{large}
\begin{figure}[h!]

\centering

\begin{tabular}{|c|}
\hline
\texttt{step} \\ \hline
1  \\ \hline
2  \\ \hline
3 \\ \hline
4 \\ \hline
\end{tabular}
\hspace{0mm}
\begin{tabular}{|c|c|c|}
\hline
\A & \B & \cellcolor{lightgray} \C \\ \hline
1 & 2 & \cellcolor{lightgray} \\ \hline
3 & & \cellcolor{lightgray} 4 \\ \hline
12 & 5 & \cellcolor{lightgray} \\ \hline
60 & & \cellcolor{lightgray} \\ \hline
\end{tabular}
\hspace{1mm}
\begin{tabular}{r}
\\
$[a_0=x_0, b_0=x_1]$  \\
\\
$[b_2 = x_2]$         \\
\\
\end{tabular}
\hspace{1em}
\begin{tabular}{l}
\\
\texttt{ADD} \\
\texttt{TIMES4} \\
\texttt{MUL} \\
\\
\end{tabular}

\caption{Execution trace modeling the computation $[(x_0+x_1)\cdot 4] \cdot x_2$ with the specific set of inputs $(1, 2, 5)$. }
\label{fig:example-execution-trace}

\end{figure}
\end{large}

Let's walk through the execution trace step by step:

\begin{enumerate}

\item First of all, we use the instruction that copies the inputs $1$ and $2$ into the columns \A and \B respectively and then we invoke the \texttt{ADD} instruction.

\item Following the \texttt{ADD} instruction, the second row of register \A now holds the sum of $1$ and $2$, resulting in $3$. At this point, we utilize the \texttt{TIMES4} instruction to obtain $3 \cdot 4 = 12$.

\item With $12$ now in the third row of register \A, we proceed to copy the third input, $5$, into register \B. The \texttt{MUL} instruction is then invoked to multiply $12$ by $5$, representing the multiplication of the current values in registers \A and \B.

\item Consequently, we obtain $60$ as the outcome of the \texttt{MUL} instruction, which is the final output for the entire computation.

\end{enumerate}

Observe that in the procedure above we have only used the available instructions proposed in our scenario. Also notice that, for any other choice of $x$, the shape of the matrix would remain the same, but most of the entries would be different. For example, the matrix in Figure \ref{fig:different-iputs-same-program} appears when using $x = (5, 3, 2)$ as input:

\begin{figure}[H]
\centering

\begin{tabular}{|c|c|c|}
\hline
\A & \B & \cellcolor{lightgray} \C \\ \hline
5 & 3 & \cellcolor{lightgray} \\ \hline
8 & & \cellcolor{lightgray} 4 \\ \hline
32 & 2 & \cellcolor{lightgray} \\ \hline
64 & & \cellcolor{lightgray} \\ \hline
\end{tabular}
\caption{The same program with other input gives the same shape but different values.}
\label{fig:different-iputs-same-program}
\end{figure}

If examine the color-coded columns in the execution trace we can identify two distinct types:

\begin{itemize}

\item \textbf{Witness columns} (colored in white): These columns depend on the input values. For example, the columns \A and \B change whenever we input different sets of values.

\item \textbf{Witness columns} (colored in gray): In contrast, these columns remain unaffected by the input values. In this example, the column \C is constant because it is only used to store the $4$ which is used always in the second step of the computation when invoking the \texttt{TIMES4} instruction. Note that fixed columns don’t change as they are an intrinsic part of the
computation.

\end{itemize}



