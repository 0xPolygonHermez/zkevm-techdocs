% !TeX spellcheck = en_US
% !TeX root = ../build/intro-proving-system.tex
% !TeX TXS-program:compile = txs:///xelatex/[--shell-escape]




\section{Polynomial Identity Language (PIL)}

In the previous section we have discussed about \textbf{execution}. However, since we are dealing with the proving system, we need some way to state the \textbf{correctness} of a given execution. The execution correctness is enforced by a set of constraints that must be fulfilled by the execution trace. This constrains are a set of identities that impose the relations between the different cells of the execution trace, determining them uniquely. Recall the previous computation example $[(x_0+x_1)\cdot 4]\cdot x_2$. In this case, the correctness of the execution trace is granted by the following set of constraints.

\vspace{1em}


\begin{figure}[ht]
\begin{minipage}[t]{0.5\linewidth}
\centering
\textbf{Program (computation)}: $[(x_0+x_1)\cdot 4]\cdot x_2$ \\
\vspace{1em}
\begin{tabular}{|c|c|c|}
\hline
\textbf{A} & \textbf{B} & \cellcolor{lightgray} \textbf{C} \\ \hline
1 & 2 & \cellcolor{lightgray} \\ \hline
3 & & \cellcolor{lightgray} 4 \\ \hline
12 & 5 & \cellcolor{lightgray} \\ \hline
60 & & \cellcolor{lightgray} \\ \hline
\end{tabular}

\caption{W}

\end{minipage}
\hfill
\begin{minipage}[t]{0.5\linewidth}
\centering
\textbf{Constraints}:
\begin{gather*}
a_0 = x_0 \\
b_0 = x_1 \\
a_1 = a_0 + b_0 \\
a_2 = a_1 \cdot 4 \\
b_2 = x_2 \\
a_3 = a_2 \cdot b_2 \\
\end{gather*}

\end{minipage}

\caption{The set of constraints (Right) for the execution trace (Left). It can be seen that if the constraints are satisfied, the program modeled using the execution trace is, indeed, correct. The reader can check that the values on the left satisfy the constraints on the right. }
\label{fig:execution-constraints}

\end{figure}

\begin{bremark}
We would like to warn the reader that thorough this document we will explain the concepts of the execution trace and the constrains in a didactic way, constructing examples and adding more complexity step by step. For this reason, some things like the way to write the constrains are not fully correct, but this will be fixed by the end of the document.
\end{bremark}

In our cryptographic back-end, each column of the matrix (equivalently, each register) is transformed it into a polynomial having degree the number of rows of the execution trace which, by performance matters, will be a power of two (in order to be possible to perform an interpolation via FFTs). This transformation is done by interpolating the values of a column over some domain (more precisely, a subgroup of roots of unity). That is, we change arithmetic constraints by polynomial constraints by means of interpolation. Constraints are actually defined among these polynomials using a language called  \textbf{PIL} (Polynomial Identity Language).

We have implemented a \href{https://github.com/0xPolygonHermez/pilcom}{PIL compiler} that reads a PIL specification file and compiles it to output a JSON file. This file contains the list of all the constraints and additional informtaion about the polyomials that can be consumed by the prover (See Figure \ref{fig:pilcom}). The repository \href{https://github.com/0xPolygonHermez/pilcom-vscode}{\texttt{pilcom-vscode}} contains a PIL syntax highlighter for VSCode.

\begin{figure}[H]
\centering
\includegraphics[width=0.7\columnwidth]{\zkevmdir/figures/architecture/intro-proving-system/pilcom-basic.drawio}
\caption{The \texttt{pilcom} compiler is in charge of compiling a file \texttt{.pil} written in PIL and obtain a \texttt{.json} file which will be consumed directly by the prover in order to generate the proof. }
\label{fig:pilcom}
\end{figure}

Putting all this together, Figure \ref{fig:generating-verifying-proofs} shows a simplified scheme that outlines the systematic process of generating and verifying proofs. The cryptographic backend receives the results of the execution (that is, both the witness and fixed columns) and the constraints, compiled using \texttt{pilcom}. The prover is in charge of generate the proof $\pi$ and the given set of publics, which are a set of values which are known both by the prover and the verifier (See Section \ref{sec:public-private-inputs} for more details). Following this, the verifier gains the capability to verify the correctness of the computation performed by the prover. The verification process involves using a specialized verification algorithm having as inputs the generated proof $\pi$ and the provided public values. The successful outcome of this verification signifies that the prover executed the computation accurately.

\begin{figure}[h]
\centering
\includegraphics[width=0.85\columnwidth]{\zkevmdir/figures/architecture/intro-proving-system/generating-verifying-proofs.drawio}
\caption{The cryptographic backend receives the execution results and constraints and generates proof $\pi$ and the corresponding public values. The verifier will use them to confirm the accuracy of the computation.}
\label{fig:generating-verifying-proofs}
\end{figure}

As we have said, a valid proof $\pi$ assures the verifier that a given computation is correct given specific public inputs. This proof is \textbf{compact} and demands minimal resources for validation. This fact enables smart contracts to act as verifiers. Our current setup relies on a back-end cryptographic system, where the ultimate verifier (we will see the meaning of this later on) is an algorithm called \textbf{FFlonk}. Practically, a smart contract in the Layer 1 implements this algorithm for the verification of the proof. The gas cost for this validation is approximately of 200,000 gas units.

\subsection{Public and private inputs} \label{sec:public-private-inputs}

Zero-Knowledge technology enables the distinction between public and private values within the cells of the execution trace, providing the capability to designate which information should be publicly available and which should remain confidential. For instance, one application involves proving knowledge of the pre-image of a hash having publicly known digest \texttt{hash\_value} while keeping the actual pre-image value \texttt{hash\_pre-image} confidential (See Figure \ref{fig:private-input-hash-example}):
\[
\texttt{hash\_value} = H(\texttt{hash\_pre-image}).
\]

\begin{figure}[h]
\centering
\includegraphics[width=0.85\columnwidth]{\zkevmdir/figures/architecture/intro-proving-system/private-input-hash-example.drawio}
\caption{Demonstrating the distinction between public and private values, this figure illustrates a hash pre-image proof. \texttt{hash\_value} is public, while \texttt{hash\_pre-image} remains confidential. The initial cell of the execution trace signifies the secret input, and through a sequence of instructions (designed to model the hash function $H$), it culminates in the production of the hash output. Subsequently, the cryptographic backend generates proof $\pi$ and public values, intended for verification by the verifier.}
\label{fig:private-input-hash-example}
\end{figure}

Publics are key important not only in the proving phase, as they are used to compute the execution trace, but also in the verifying phase. As shown in Figure \ref{fig:generating-verifying-proofs}, they are given to the verifier together with the proof so the verifier can actually verify the computation done by the prover. By default, all values in PIL are considered private. However, you can specify a particular value to be public by using the keyword \texttt{public}.

%Here we have two execution trace for the same computation and the same inputs but with different privacy in the inputs.

%\begin{multicols}{2}
%\centering
%$$
%\begin{array}{|c|c|c|}
%\hline
%\textbf{A} & \textbf{B} & \cellcolor{lightgray} \textbf{C} \\ \hline
%1 \cellcolor{green} & 2 \cellcolor{green} & \cellcolor{lightgray} \\ \hline
%3 & & 4 \cellcolor{lightgray} \\ \hline
%12 & 5 \cellcolor{green} & \cellcolor{lightgray} \\ \hline
%60 \cellcolor{green} & & \cellcolor{lightgray} \\ \hline
%\end{array}
%$$
%
%\columnbreak
%
%$$
%\begin{array}{|c|c|c|}
%\hline
%\textbf{A} & \textbf{B} & \cellcolor{lightgray} \textbf{C} \\ \hline
%1 \cellcolor{green} & 2 \cellcolor{orange} & \cellcolor{lightgray} \\ \hline
%3 & & 4 \cellcolor{lightgray} \\ \hline
%12 & 5 \cellcolor{green} & \cellcolor{lightgray} \\ \hline
%60 \cellcolor{green} & & \cellcolor{lightgray} \\ \hline
%\end{array}
%$$
%\end{multicols}
%
%$\begin{array}{|c|}
%\cellcolor{green}{public}\\ \hline
%
%
%\cellcolor{orange}{private}\\ \hline
%
%
%\cellcolor{lightgray}{fixed}\\ \hline
%\end{array}$



