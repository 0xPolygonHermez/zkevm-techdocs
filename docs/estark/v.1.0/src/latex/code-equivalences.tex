
\section{Code Naming Equivalences}

Aiming for clear equations and explanations, we prefer not to use the same names appearing in code. Henceforth, in this section, we show several equivalence tables in order to make the reader easier to overcome the problem of reading different names between the code and the technical documentation. 

\begin{figure}[H]
\[
\begin{array}{|c|c|}
\hline
\textbf{Technical Documentation}	&\textbf{Code}  	\\ \hline
\alpha								&\texttt{u}			\\  
\beta								&\texttt{defVal}	\\
\gamma								&\texttt{gamma}		\\
\delta								&\texttt{betta}		\\
\afr								&\texttt{vc}		\\
z									&\texttt{xi}		\\
\epsilon_1							&\texttt{v1}		\\
\epsilon_2							&\texttt{v2}		\\
\hline
\end{array}
\]
\caption{Challenges Naming Equivalences. }
\end{figure}




\begin{figure}[H]
\[
\begin{array}{|c|c|}
\hline
\textbf{Technical Documentation}	&\textbf{Code}  	\\ \hline
Q(X) := Q_1(X) + X^n \cdot Q_2(X)	&\texttt{q}			\\  
Q_1(X)								&\texttt{qq1}		\\
Q_2(X)								&\texttt{qq2}		\\
\FRI(X)								&\texttt{friPol}	\\
\hline
\end{array}
\]
\caption{Polynomials Naming Equivalences. }
\end{figure}